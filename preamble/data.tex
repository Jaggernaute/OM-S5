%% Thesis type: bachelor, master, doctoral
\NewExpandableDocumentCommand{\ThesisType}{}{Licence 3 EEEA}

%% Thesis title (exactly as in the formal assignment)
\NewExpandableDocumentCommand{\ThesisTitle}{}{Outils mathématiques}
%% Plaintext version for PDF metadata, uncomment if needed (defauls to \ThesisTitle)
\NewExpandableDocumentCommand{\ThesisTitlePlaintext}{}{Outils mathématiques}
%% Thesis title (if custom formatting is needed for the title page, defauls to \ThesisTitle)
\NewExpandableDocumentCommand{\ThesisTitleFront}{}{
    \ThesisTitle\\
    {\Huge\color{gray}vPre 0.0.1}
}

%% Author of the thesis
\NewExpandableDocumentCommand{\ThesisAuthor}{}{Alex Videcoq}
%% Plaintext version for PDF metadata, uncomment if needed (defauls to \ThesisAuthor)
\NewExpandableDocumentCommand{\ThesisAuthorPlaintext}{}{Alex Videcoq}

%% Year when the thesis is submitted
\NewDocumentCommand{\YearSubmitted}{}{2025}
%% Year of the last revision, uncomment if it is different from \YearSubmitted
% \NewDocumentCommand{\YearRevision}{}{2025}

%% University
\NewDocumentCommand{\University}{}{Université de Rennes}

%% Name of the department or institute, where the work was officially assigned
\NewDocumentCommand{\Department}{}{ISTIC}

%% Is it a Department (katedra), or an Institute (ústav)?
\NewDocumentCommand{\DeptType}{}{UFR}

%% Thesis supervisor: name, surname and titles
\NewDocumentCommand{\Supervisor}{}{Prof. John Doe}
%% Thesis co-supervisor: name, surname and titles (uncomment if applicable)
% \NewDocumentCommand{\CoSupervisor}{}{Prof. Matthieu Davy}

%% Supervisor's department/institute
\NewDocumentCommand{\SupervisorsDepartment}{}{ISTIC}

%% Study programme and specialization
\NewDocumentCommand{\StudyProgramme}{}{L3 EEEA}

%% Abstract (recommended length around 80-200 words; this is not a copy of your thesis assignment!)
\NewDocumentCommand{\Abstract}{}{
\NewDocumentCommand{\Abstract}{}{
    Ce travail présente les notions fondamentales liées aux espaces vectoriels de fonctions,
    à la définition des normes et des distances, ainsi qu’aux différents types de convergence
    des suites de fonctions. Sont également abordés le rôle du produit scalaire, la norme
    hermitienne et les propriétés des systèmes orthogonaux. L’accent est mis sur les définitions,
    propriétés essentielles et exemples permettant d’illustrer ces concepts en analyse
    fonctionnelle.
}
}
%% Subject (short description for PDF metadata)
\NewExpandableDocumentCommand{\Subject}{}{%
    Analyse fonctionnelle et outils mathématiques pour espaces vectoriels
}

%% Keywords (about 3-7)
\NewExpandableDocumentCommand{\Keywords}{}{%
    Espaces vectoriels, Normes, Convergence, Produit scalaire, Orthogonalité, Analyse fonctionnelle
}
%% Plaintext version for PDF metadata, uncomment if needed (defauls to \Keywords)
\NewExpandableDocumentCommand{\KeywordsPlaintext}{}{%
    Espaces vectoriels, Normes, Convergence, Produit scalaire, Orthogonalité, Analyse fonctionnelle
}
