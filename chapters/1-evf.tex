\chapter{Espaces vectoriels de fonctions}\label{chap:evf}

Dans ce chapitre, les espaces vectoriels considérés sont construits
avec des scalaires qui sont, soit des nombres réels, soit des nombres complexes.
L’ensemble des scalaires utilisés seront notés \(\mathbb{K}\). On préciseras
\(\mathbb{K} = \R\) ou bien \(\mathbb{K} = \C\).

\begin{definition}{Espace vectoriel de polynômes du premier degré}{}
  On note
  \[
    \mathcal{P}_1(\mathbb{K}) = \{\, a_0 + a_1x \mid a_0, a_1 \in
    \mathbb{K}\,\}.
  \]
  L’ensemble $\mathcal{P}_1(\mathbb{K})$, muni de l’addition des
  polynômes et de la multiplication externe par un scalaire,
  forme un espace vectoriel noté
  \[
    \bigl(\mathcal{P}_1(\mathbb{K}), +, \cdot \bigr).
  \]

\begin{example}{Base et dimension}
  La famille $\{1, x\}$ constitue une base canonique de
  $\mathcal{P}_1(\mathbb{K})$.
  Ainsi, $\dim \mathcal{P}_1(\mathbb{K}) = 2$.
\end{example}
\end{definition}

\begin{Note}
  Aussi abstraite que paraisse cette idée : voir des fonctions comme
  des vecteurs(des points) d’un espace vectoriel, elle est en réalité d’une
  extraordinaire portée. C’est le point de départ de théories et de techniques
  utiles bien au-delà des mathématiques\par
  \emph{Cours de M.RUMIN réécrit par J.KULCSAR, S2PMCP, Université Paris-Saclay}
\end{Note}
