\chapter{Espaces vectoriels de fonctions} \label{chap:evf}

\section{Rappel sur les espaces vectoriels}
Un \emph{espace vectoriel de fonctions} est un espace dont les
vecteurs sont des fonctions. Le concept d'espace vectoriel étant en
fait de \emph{créer une structure mathématique abstraite} permettant de
faire des calculs géométriques dans des ensembles qui n'ont a priori
pas vocation à être appréhendés de façon géométrique, comme des
ensembles de \textbf{matrices ou de fonctions}.

\section{Introduction}
Dans ce chapitre, les espaces vectoriels considérés sont construits
sur un corps quelconque not\'e $\mathbb{K}$, mais en pratique on aura
\textbf{toujours} $\mathbb{K}=\mathbb{R}$ ou $\mathbb{K}=\mathbb{C}$.

\begin{Note}
	Pour aller beaucoup plus loin sur les structures alg\'ebriques :
	\begin{itemize}
		\item \href{https://rancune.org/sciences/2022/03/23/structures-algebriques.html}{Structures algébriques : THE poster}
		\item \emph{Les maths en tête : Algèbre} par \textbf{Xavier Gourdon}\Cite{Gourdon-algebre}
	\end{itemize}
\end{Note}

\section{Exemples}

\subsection{Polynômes de degré $\leq n$}
On note~:
\[
  F_n = \Bigl\{\, P(x) = a_0 + a_1 x + a_2 x^2 + \cdots + a_n x^n
  \;\Bigm|\; a_i \in \mathbb{K},\; n \geq 1 \,\Bigr\}.
\]
Muni de l’addition des fonctions et de la multiplication externe par
un scalaire,
$(F_n,+,\cdot)$ est un espace vectoriel de dimension $n+1$.

\medskip
La base canonique de $F_n$ est
\[
  \mathcal{B}_n = \{1, x, x^2, \dots, x^n\}.
\]

Soient $P(x)=\sum_{i=0}^n a_i x^i$ et $Q(x)=\sum_{i=0}^n b_i x^i$ :
\begin{itemize}
  \item \textbf{Addition :}
    \[
        +:\begin{cases}E\times E\rightarrow E\\(u,v)\mapsto u+v\end{cases},\quad
        (P+Q)(x) = \sum_{i=0}^n (a_i+b_i)x^i \in F_n.
    \]

  \item \textbf{Multiplication par un scalaire :} pour $\alpha \in \mathbb{K}$,
    \[
        \cdot:\begin{cases}\mathbb{K}\times E\rightarrow E\\(\lambda,u)\mapsto \lambda u\end{cases},\quad
        (\alpha P)(x) = \sum_{i=0}^n (\alpha a_i)x^i \in F_n.
    \]
\end{itemize}

\subsubsection{Cas particulier $n=2$}
Si $n=2$, tout polynôme s’écrit
\[
  P(x) = a_0 + a_1x + a_2x^2.
\]
L’espace $F_2$ est de dimension $3$, de base $(1,x,x^2)$, et est en
bijection naturelle avec
$\mathbb{K}^3$ via
\[
  P(x) \longmapsto (a_0,a_1,a_2).
\]

\subsection{Séries trigonométriques}
On considère l’ensemble
\[
  G_I = \Bigl\{\, f(x) = \sum_{n\in I} \alpha_n e^{inx}
  \;\Bigm|\; I \subset \mathbb{Z},\; \alpha_n \in [a,b] \,\Bigr\}.
\]
Pour $f,g \in G_I$ et $\lambda \in \mathbb{K}$ :
\[
  (f+g)(x) = f(x)+g(x),
  \qquad
  (\lambda f)(x) = \lambda f(x).
\]
Ainsi, $(G_I,+,\cdot)$ est également un espace vectoriel.
