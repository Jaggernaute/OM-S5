\chapter{Normes et distances dans les espaces vectoriels} \label{chap:norme}

\section{Espaces vectoriels norm\'es}

\begin{Note}
Le d\'eveloppement des espaces vectoriels norm\'es (en particulier de dimensions
infinies) est d’abord d\^u \`a Hilbert ; Banach compl\'eta largement cette
th\'eorie dans les ann\'ees 1930.
\end{Note}

Une \emph{norme} sur un \(\mathbb{K}\)-espace vectoriel \(E\) (sur \(\mathbb{R}\) ou \(\mathbb{C}\)) est une application
\[
E \to \mathbb{R}^+,\quad u \mapsto \|u\|
\]
qui associe \`a chaque vecteur \(u\in E\) un nombre r\'eel positif, not\'e \(\|u\|\), et v\'erifie les trois propri\'et\'es fondamentales :
\begin{itemize}
  \item (s\'eparation) \(\|u\|=0 \iff u=0\) : seule la longueur du vecteur nul est nulle ;
  \item (homog\'en\'eit\'e) Pour tout \(\lambda\in\mathbb{K}\), pour tout \(x\in E\), \(\|\lambda x\|=|\lambda|\cdot\|x\|\) ;
  \item (in\'egalit\'e triangulaire) Pour tout \((u,v)\in E^2\), \(\|u+v\|\le \|u\|+\|v\|\).
\end{itemize}

Muni d’une norme, \(E\) est appel\'e un \(\mathbb{K}\)-espace vectoriel norm\'e (en abr\'eg\'e e.v.n).

\begin{Note}
    \emph{Les maths en t\^ete : Analyse} \Cite{Gourdon-analyse} \textit{Topologie sur les espaces m\'etriques
    et les espaces vectoriels norm\'es}, \texttt{1.1} Normes et Distances.
\end{Note}

Ainsi, un \emph{espace vectoriel norm\'e} est un espace vectoriel muni d’une telle norme.
Intuitivement, la norme \(\|u\|\) repr\'esente la «longueur» du vecteur \(u\) dans l’espace.
L’in\'egalit\'e triangulaire traduit la propri\'et\'e g\'eom\'etrique selon laquelle le chemin direct
est plus court que tout chemin bris\'e par un point interm\'ediaire.

On appelle aussi \emph{boule unit\'e} l’ensemble
\[
B(0,1) = \{x\in E : \|x\|\le 1\}.
\]

\section{Exemples de normes}

\subsection{Normes dans \(\mathbb{R}^n\)}

Dans \(\left(\mathbb{R}^n,+,\cdot\right)\), plusieurs normes usuelles sont d\'efinies par des formules explicites.
Pour \(x=(x_1,\dots,x_n)\in\mathbb{R}^n\) on d\'efinit :
\[
\|x\|_2 = \sqrt{x_1^2+\cdots+x_n^2},
\qquad
\|x\|_1 = \sum_{i=1}^n |x_i|,
\qquad
\|x\|_\infty = \max_{1\le i\le n}|x_i|.
\]

Ces trois applications satisfont bien les axiomes d’une norme.
La norme \(\|\cdot\|_2\) est la norme euclidienne usuelle.
Plus g\'en\'eralement, pour tout \(\alpha\in[1,\infty)\) on d\'efinit la \emph{norme \(\ell^\alpha\)} par :
\[
\|x\|_\alpha = \Bigl(\sum_{i=1}^n |x_i|^\alpha\Bigr)^{1/\alpha}.
\]
C’est une cons\'equence de l’in\'egalit\'e de Minkowski.

\subsection{Espaces \(\ell^\alpha\) de suites sommables}

Pour \(1\le \alpha<\infty\), l’espace \(\ell^\alpha\) est l’ensemble des suites \(x=(x_n)_{n\ge1}\) telles que
\[
\sum_{n=1}^\infty |x_n|^\alpha < \infty.
\]
On y d\'efinit la norme
\[
\|x\|_\alpha = \Bigl(\sum_{n=1}^\infty |x_n|^\alpha\Bigr)^{1/\alpha}.
\]

Pour \(\alpha=2\), l’espace \((\ell^2,\|\cdot\|_2)\) est un espace de Hilbert muni du produit scalaire usuel.
Ces espaces \((\ell^\alpha,\|\cdot\|_\alpha)\) sont des espaces de Banach, c’est-\`a-dire complets pour la distance associ\'ee \`a la norme.

\subsection{Fonctions continues sur \([a,b]\)}

Soit \(C([a,b])\) l’espace vectoriel des fonctions continues \(f:[a,b]\to\mathbb{K}\)
(avec \(\mathbb{K}=\mathbb{R}\) ou \(\mathbb{C}\)).
On y d\'efinit plusieurs normes classiques :
\[
\|f\|_\alpha = \Bigl(\int_a^b |f(x)|^\alpha\,dx\Bigr)^{1/\alpha}, \qquad 1\le \alpha<\infty,
\]
et la norme du \emph{sup} :
\[
\|f\|_\infty = \sup_{x\in[a,b]} |f(x)|.
\]

La norme \(\|f\|_\alpha\) mesure la « taille moyenne » de la fonction,
tandis que \(\|f\|_\infty\) mesure la valeur maximale atteinte par la fonction.
\((C([a,b]),\|\cdot\|_\infty)\) est un espace de Banach.

\subsection{Espaces \(L^\alpha\) et \'egalit\'e presque partout}

Dans un cadre plus g\'en\'eral, on consid\`ere un espace mesur\'e \((X,\mu)\).
On d\'efinit \(L^\alpha(X)\) comme l’ensemble des fonctions mesurables \(f\) telles que
\[
\int_X |f(x)|^\alpha \, d\mu(x) < \infty.
\]

On identifie deux fonctions \(f\) et \(g\) si \(f(x)=g(x)\) pour presque tout \(x\) (c’est-\`a-dire sauf sur un ensemble de mesure nulle).
On pose alors
\[
\|f\|_\alpha = \Bigl(\int_X |f(x)|^\alpha\,d\mu(x)\Bigr)^{1/\alpha}.
\]

Ces espaces sont norm\'es, complets pour \(\alpha\ge 1\).
En particulier, \(L^1\) est l’espace des fonctions absolument int\'egrables
et \(L^2\) celui des fonctions de carr\'e int\'egrable.

\section{Distances et espaces m\'etriques}

\subsection{D\'efinition}

Une \emph{distance} (ou \emph{m\'etrique}) sur un ensemble \(E\) est une application
\[
d:E\times E\to [0,+\infty[
\]
v\'erifiant, pour tous \(x,y,z\in E\) :
\begin{itemize}
  \item \(d(x,y)=0 \iff x=y\) ;
  \item \(d(x,y)=d(y,x)\) (sym\'etrie) ;
  \item \(d(x,z)\le d(x,y)+d(y,z)\) (in\'egalit\'e triangulaire).
\end{itemize}

Un tel couple \((E,d)\) est appel\'e \emph{espace m\'etrique}.

\subsection{Distance associ\'ee \`a une norme}

Si \(E\) est un espace vectoriel norm\'e, on d\'efinit la distance induite par la norme :
\[
d(u,v) = \|u-v\|.
\]

Cette distance satisfait les axiomes pr\'ec\'edents et repr\'esente la longueur du segment joignant \(u\) \`a \(v\).
Gr\^ace \`a \(d\), on d\'efinit les boules ouvertes
\[
B(u,r) = \{x\in E : \|x-u\|<r\},
\]
ce qui induit une topologie naturelle sur \(E\).
Ainsi, tout espace vectoriel norm\'e est aussi un espace m\'etrique.
