% \setcounter{chapter}{16} % "Q" (17th letter, but latex first
% increments -> 17-1=16)
% \chapter{Résumé Rapide}
\chapternotnumbered{Résumé Rapide} \label{ch:Quick Summary}

\vspace{2ex} % espace vertical supplémentaire, car la lettre Q a une
% longue queue

\textbf{Espaces vectoriels de fonctions \Cref{chap:evf}} : Définit la
structure des espaces vectoriels appliquée aux fonctions,
polynômes, séries trigonométriques, suites et fonctions continues.

\textbf{Normes et espaces vectoriels normés \Cref{chap:norme}} :
Introduit la notion de norme, présente les normes usuelles ($\ell^1$,
$\ell^2$, norme du sup) et leurs applications aux suites et aux fonctions.

\textbf{Distances et métriques \Cref{chap:distance}} : Décrit la
définition générale d’une distance, ses propriétés, et la distance
induite par une norme dans un espace vectoriel.

\textbf{Convergences de suites de fonctions \Cref{chap:convergeance}}
: Distingue convergence simple, convergence en moyenne ($L^1$),
convergence en moyenne quadratique ($L^2$) et convergence uniforme,
en explicitant leurs relations.

\textbf{Produit scalaire et norme hermitienne
\Cref{chap:produit-scalaire}} : Définit le produit scalaire, expose
l’inégalité de Cauchy-Schwarz et introduit la norme hermitienne.

\textbf{Systèmes orthogonaux \Cref{chap:systemes}} : Présente les
notions d’orthogonalité et d’orthonormalité, ainsi que les propriétés
fondamentales des sous-espaces orthogonaux et l’extension du théorème
de Pythagore.
